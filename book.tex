\documentclass[envcountsame,envcountchap]{svmono}


\usepackage{makeidx}                % allows index generation
\usepackage{graphicx}               % standard LaTeX graphics tool
%\usepackage{xtab}                                   % when including figure files
\usepackage{multicol}
\usepackage{qtree}
\usepackage{booktabs}
\usepackage{latexsym}
\usepackage{psfig}                  % used for the two-column index
\usepackage[bottom]{footmisc}       % places footnotes at page bottom
\usepackage[italian]{babel}
\usepackage{blindtext}

\usepackage{my_listings}
\usepackage[hyphens]{url}
\usepackage{epstopdf}

\usepackage[ansinew]{inputenc}

%\setlength{\textwidth}{12.4cm} \setlength{\textheight}{18.0cm}
%\setlength{\oddsidemargin}{4.60cm}
%\setlength {\evensidemargin}{0.5cm}  %foglio a destra 0.06
%\setlength {\topmargin}{-1cm}
%
%\hoffset -8mm

\setlength{\textwidth}{12.7cm}
\setlength{\textheight}{20.0cm}
\setlength{\oddsidemargin}{3.50cm}
\setlength {\evensidemargin}{-0.3cm}  %foglio a destra 0.06
\setlength {\topmargin}{-1cm}

\renewcommand\definitionname{Definizione}
\renewcommand\theoremname{Teorema}
\renewcommand\propositionname{Proposizione}
\renewcommand\corollaryname{Corollario}
\renewcommand\examplename{Esempio}


\lstnewenvironment{codice}[1][Listato]
{\lstset{basicstyle=\small\ttfamily, columns=fullflexible,
		keywordstyle=\color{red}\bfseries, commentstyle=\color{blue},
		language=Java, basicstyle=\small,
		frame=shadowbox, float=*, #1}}{}





\def\punto{\hspace*{\fill}\Box}

\catcode`\�=\active \def �{\`{\i}} \catcode`\�=\active \def
�{\`{\i}} \catcode`\�=\active \def �{\`e} \catcode`\�=\active \def
�{\`e} \catcode`\�=\active \def �{\`E} \catcode`\�=\active \def
�{\`E} \catcode`\�=\active \def �{\`a} \catcode`\�=\active \def
�{\`a} \catcode`\�=\active \def �{\`A} \catcode`\�=\active \def
�{\`A} \catcode`\�=\active \def �{\`u} \catcode`\�=\active \def
�{\`u} \catcode`\�=\active \def �{\`o} \catcode`\�=\active \def
�{\`o}


\makeindex             % used for the subject index
                       % please use the style svind.ist with
                       % your makeindex program

%%%%%%%%%%%%%%%%%%%%%%%%%%%%%%%%%%%%%%%%%%%%%%%%%%%%%%%%%%%%%%%%%%%%%
\date{}
\begin{document}

%\maketitle


\pagenumbering{Roman}
%\setcounter{page}{5}

\frontmatter

\begin{titlepage}

    \begin{center}

    \large{\bf Universit� degli Studi Mediterranea di Reggio Calabria}

    \vspace*{1mm}

    \large{Dipartimento di Ingegneria dell'Informazione, delle Infrastrutture e dell'Energia Sostenibile}

    \vspace*{1mm}

    \normalsize{Corso di Laurea in Ingegneria dell'Informazione}

    \vspace*{1mm}

    %\normalsize{Dipartimento di Informatica, Matematica, Elettronica e Trasporti}

    \hspace*{-0mm}

    \rule{125mm}{.2mm}  %{lunghezza}{spessore}


    \vspace{18mm}

    \begin{figure}[h!]
    \centerline{\includegraphics[width=2cm]{logounirc.png}}
    \end{figure}

    \vspace{5mm}

    \textbf{Tesi di Laurea}

    \vspace{5mm}

    \large{\bf TITOLO DELLA TESI }

    \vspace{22mm}

    \begin{tabular}{lcl}
    {\large Relatore} & \ \hskip 4cm \ & {\large Candidato} \\
    \ & \ & \ \\
    {Prof. Francesco Verdi} &    & {Andrea Rossi}\\
    \ & \ & \ \\
    \ & \ & \ \\
    \\
    \end{tabular}

    \rule{125mm}{.2mm}

    \textbf{Anno Accademico 2017-2018}
    \end{center}

\end{titlepage}



\newpage


\tableofcontents

\listoffigures



\mainmatter%%%%%%%%%%%%%%%%%%%%%%%%%%%%%%%%%%%%%%%%%%%%%%%%%%%%%%%

\pagenumbering{arabic} \setcounter{page}{1}

\chapter{Introduzione} \label{introduzione}

Esempio di tesi (prodotta con \LaTeX)

\vspace{2cm}

\Blindtext[3]





\chapter{Questo � il capitolo 2} \label{cap2}
\begin{flushright}
\textit{E' possibile inserire qui una sintesi breve del contenuto dell'intero capitolo}
\end{flushright}
\medskip


\Blindtext[3]






\section{Esempio di sezione}\label{sec:analisi_requisiti}
\Blindtext[3]

\subsection{Esempio di sotto-sezione}\label{sec:diagrammi}


\begin{figure}[t]
\centerline{\includegraphics[width=10cm]{logounirc.png}}
\caption{Esempio di figura}\label{fig:progetto}
\end{figure}

Esempio di riferimento a figura: in Figura \ref{fig:progetto} � mostrato ... 



\begin{table}[t]
    \begin{center}
    \begin{tabular}{||c|l||}
      \hline
      \hline
     Col 1  & \ Col 2 \\ \hline
      xxx  & \ ??? \\ \hline
      xxx  & \ ??? \\ \hline
      xxx & \ ??? \\ \hline
    \hline
    \hline
    \end{tabular}
    \end{center}
    \caption{Esempio di tabella}\label{table:notation}
\end{table}


Esempio di riferimento a tabella: in Tabella \ref{table:notation} ....


\subsection{Un altro esempio di sotto-sezione}
\Blindtext[5]

\subsection{Ultimo esempio di sotto-sezione}
\Blindtext[5]




\chapter{Conclusioni}
\label{cap:conclusioni}


\Blindtext[6]






\begin{thebibliography}{1}
	
	\bibitem{Android}
	{Sito UNIRC}.
	\newblock {\em \tt \url{http://www.unirc.it}}, 2018.
	
	\bibitem{vidas2011all}
	Sito DIIES.
	\newblock {\em \tt \url{http://www.diies.unirc.it}}, 2018.
	
\end{thebibliography}


\end{document} 