\documentclass[envcountsame,envcountchap]{svmono}
% Import fontspec package for font management
\usepackage{fontspec}

% Set main font (example using "Times New Roman")
%\setmainfont{Baskerville}

\usepackage{parskip}
\usepackage{makeidx}    % allows index generation
\usepackage{graphicx}   % standard LaTeX graphics tool when including figure files
\usepackage{multicol}
\usepackage{booktabs}
\usepackage{latexsym}
\usepackage[italian]{babel}
\usepackage{blindtext}
\usepackage{xcolor}
\usepackage{multirow}
\usepackage{multicol}
\usepackage{float}
\usepackage{minted}
\usepackage[hyphens]{url}
\usepackage{acronym}
\usepackage{natbib}
\usepackage{array}
\usepackage{lscape}
\usepackage{hyperref} % serve per inserire link nel testo, 
% utile solo per documenti non destinati alla stampa


\setlength{\textwidth}{12.7cm}
\setlength{\textheight}{20.0cm}
\setlength{\oddsidemargin}{3.50cm}
\setlength {\evensidemargin}{-0.3cm}  %foglio a destra 0.06
\setlength {\topmargin}{-1cm}

\renewcommand\definitionname{Definizione}
\renewcommand\theoremname{Teorema}
\renewcommand\propositionname{Proposizione}
\renewcommand\corollaryname{Corollario}
\renewcommand\examplename{Esempio}




\def\punto{\hspace*{\fill}\Box}


\makeindex             % used for the subject index
                       % please use the style svind.ist with
                       % your makeindex program

%%%%%%%%%%%%%%%%%%%%%%%%%%%%%%%%%%%%%%%%%%%%%%%%%%%%%%%%%%%%%%%%%%%%%
\date{}
\begin{document}

%\maketitle


\pagenumbering{Roman}
%\setcounter{page}{5}

\frontmatter

\begin{titlepage}

    \begin{center}

    \large{\bf Università degli Studi Mediterranea di Reggio Calabria}

    \vspace*{1mm}

    \large{Dipartimento di Ingegneria Civile, Energia, Ambiente e Materiali}

    \vspace*{1mm}

    \normalsize{Corso di Laurea in Ingegneria Industriale}

    \vspace*{1mm}

    \hspace*{-0mm}

    \rule{125mm}{.2mm}  %{lunghezza}{spessore}


    \vspace{18mm}

    \begin{figure}[h!]
        \centerline{\includegraphics[width=3cm]{logounirc.png}}
    \end{figure}

    \vspace{5mm}

    \textbf{Tesi di Laurea}

    \vspace{5mm}

    \large{\bf <TITOLO DELLA TESI> }

    \vspace{22mm}

    \begin{tabular}{lcl}
        {\large Relatore} & \ \hskip 2.2cm \ & {\large Candidato} \\
        \ & \ & \ \\
        {<NOME RELATORE>} &                  & {<NOME CANDIDATO>}\\
        \ & \ & \ \\
        {\large Correlatore} &               & \\ %commentare se assente
        \ & \ & \ \\
        {<NOME CORRELATORE>} &               & \\ %commentare se assente
        \\
    \end{tabular}

    \rule{125mm}{.2mm}

    \textbf{Anno Accademico 2023-2024}
    \end{center}

\end{titlepage}

\newpage
\newpage
\cleardoublepage
\thispagestyle{empty}
\vspace*{\stretch{1}}
\begin{flushright}
\itshape
<DEDICA>
\end{flushright}
\vspace{\stretch{2}}
\cleardoublepage

\newpage

\thispagestyle{empty}

% questo aggiunge l'indice al documento
\tableofcontents

\listoffigures



\pagenumbering{arabic} \setcounter{page}{7}

\chapter*{Indice degli acronimi}
\markboth{Indice degli acronimi}{Indice degli acronimi}
\begin{acronym}[WYSIWYM]
    % gli acronimi devono essere ordinati manualmente
    \acro{DICEAM}{Dipartimento di Ingegneria Civile, dell'Energia, dell'Ambiente e dei Materiali}
    \acro{DIIES}{Dipartimento di Ingegneria dell'Informazione, delle Infrastrutture e dell'Energia Sostenibile}
    \acro{DiGiES}{Dipartimento di Giurisprudenza, Economia e Scienze Umane}
    \acro{dArTe}{Dipartimento di Architettura e Territorio}
    \acro{PAU}{Dipartimento di Patrimonio, Architettura e Urbanistica}
\end{acronym}




\chapter*{Introduzione} \label{introduzione}
\addcontentsline{toc}{chapter}{Introduzione}
\markboth{Introduzione}{Introduzione}

\vspace{2cm}

\hfill \textit{<METTI QUÌ UNA BREVE PRESENTAZIONE DEL CAPITOLO>}

\vspace{0.5cm}


Contenuto del capitolo.
Se vai a capo una volta sola, \LaTeX non va a capo.

Così invece sì.

Oh, guarda, un elenco puntato:
\begin{itemize}
    \item Questo
    \item È
    \item Un
    \item Elenco
    \item Puntato
\end{itemize}




\mainmatter%%%%%%%%%%%%%%%%%%%%%%%%%%%%%%%%%%%%%%%%%%%%%%%%%%%%%%%


\chapter{<TITOLO DEL PRIMO CAPITOLO>} \label{Cap.1}

\vspace{2cm}

\begin{flushright}
\textit{<ALTRA BREVE PRESENTAZIONE>}
\end{flushright}

\vspace{0.5cm}

\section{<TITOLO DELLA SEZIONE>}
\LaTeX permette di creare sezioni, sottosezioni e sotto-sottosezioni.

\subsection{<TITOLO DELLA SOTTOSEZIONE>}
Questa è una sottosezione.

\subsubsection{<TITOLO DELLA SOTTO-SOTTOSEZIONE>}
Ed ecco la sotto-sottosezione.

Non ti sono bastati 3 livelli?

\href{https://tex.stackexchange.com/questions/60209/how-to-add-an-extra-level-of-sections-with-headings-below-subsubsection}{Questo è un link per stackexchange, spiega come aggiungere altri livelli di sottosezioni.}

\section{È il momento delle figure}

\LaTeX adora spostare le figure, quindi bisognerà un po' giocare col testo per trovare la posizione adatta.


\begin{figure}[h]
    \centering 
    \includegraphics[width=0.2\linewidth]{logodiceam.png} 
    \caption{Per comodità useremo il logo del dipartimento DICEAM.}
    \label{figura_logo_diceam}
\end{figure}

In generale è consigliabile lasciar fare tutto al compilatore e poi referenziare la figura così \ref{figura_logo_diceam}. \citep{forceFigurePlacement}

Se proprio è necessario posizionare l'immagine in una posizione precisa, usare [H] invece di [h].

\section{Syntax highlighting}
Se a qualcuno piacesse scrivere codice e magari renderlo leggibile, minted fa al caso suo.
Altri stili: \\
\url{https://www.overleaf.com/learn/latex/Code_Highlighting_with_minted}

\begin{minted}{python}
import numpy as np
    
def incmatrix(genl1,genl2):
    m = len(genl1)
    n = len(genl2)
    M = None #to become the incidence matrix
    VT = np.zeros((n*m,1), int)  #dummy variable
    
    #compute the bitwise xor matrix
    M1 = bitxormatrix(genl1)
    M2 = np.triu(bitxormatrix(genl2),1) 

    for i in range(m-1):
        for j in range(i+1, m):
            [r,c] = np.where(M2 == M1[i,j])
            for k in range(len(r)):
                VT[(i)*n + r[k]] = 1;
                VT[(i)*n + c[k]] = 1;
                VT[(j)*n + r[k]] = 1;
                VT[(j)*n + c[k]] = 1;
                
                if M is None:
                    M = np.copy(VT)
                else:
                    M = np.concatenate((M, VT), 1)
                
                VT = np.zeros((n*m,1), int)
    
    return M
\end{minted}

\chapter{<SECONDO CAPITOLO>} \label{Cap.2}

\vspace{2cm}

\begin{flushright}
\textit{<ALTRA BREVE DESCRIZIONE>}
\end{flushright}

\vspace{0.5cm}

Testo multicolonna: 

\begin{multicols}{2}
    \textbf{Colonna 1}\\
    Testo della prima colonna.
\columnbreak\\
    \textbf{Colonna 2}\\
    Non lo scrivo nemmeno.
\end{multicols}


% \chapter{<TERZO CAPITOLO>}  \label{Cap.3}

% \vspace{2cm}

% \begin{flushright}
%  \textit{<<>>}
% \end{flushright}

% \vspace{0.5cm}

Signore e signori, sua maestà l'elenco numerato:
\begin{enumerate}
    \item Lorem
    \item Ipsum
    \item Dolor
    \item Sit
    \item Amet
    \item Sì ho avuto molta fantasia
\end{enumerate} 

%questo abominio serve per forzare latex a saltare una riga
~\\

Un elenco di elenchi??

Roba degna del più forte elencatore d'Italia. \citep{elencatoreSeriale}
\begin{itemize}
    \item \textbf{A}: AAAA
    \begin{itemize}
        \item Aa
        \item Ab
        \item Ac
        \item Ad
        \item Ae
        \item Af
    \end{itemize} 
    \item \textbf{B}: BBBB
    \begin{itemize}
        \item Ba
        \item Bb
        \item Bc
        \item Bd
        \item Be
        \item Bf
    \end{itemize}
\end{itemize}

\newpage
newpage termina la pagina attuale e inizia una nuova pagina.

\vspace*{2cm}

Non avevamo ancora parlato delle tabelle.

Sono lunghe da trattare quindi lasciamo il duro compito alla documentazione di Overleaf.
\url{https://www.overleaf.com/learn/latex/Tables}

\begin{table}[h!]
\centering
\begin{tabular}{||c c c c||} 
    \hline
    Col1 & Col2 & Col2 & Col3 \\ [0.5ex] 
    \hline\hline
    1 & 6 & 87837 & 787 \\ 
    2 & 7 & 78 & 5415 \\
    3 & 545 & 778 & 7507 \\
    4 & 545 & 18744 & 7560 \\
    5 & 88 & 788 & 6344 \\ [1ex] 
    \hline
\end{tabular}
\caption{Table to test captions and labels.}
\label{table:1}
\end{table}


\newpage
\begin{landscape}
landscape mette il contenuto della pagina in orizzontale.
\end{landscape}
    


\chapter*{Conclusioni}
\addcontentsline{toc}{chapter}{Conclusioni}
\markboth{Conclusioni}{Conclusioni}

\vspace{2cm}

\begin{flushright}
 \textit{<SINTESI DELLE CONCLUSIONI>}
\end{flushright}

\vspace{0.5cm}

Le conclusioni funzionano come tutti gli altri capitroli.

Come l'introduzione questo capitolo non ha il numero.

\section*{Sezione delle conclusioni}
La sezione con l'asterisco non ha il numero.

\chapter*{Ringraziamenti}
\addcontentsline{toc}{chapter}{Ringraziamenti}
\markboth{Ringraziamenti}{Ringraziamenti}

Ringrazio il dipartimento DIIES per aver pubblicato un modello di tesi non troppo aggiornato.

Ringrazio Alessandro Campolo\footnote{``ringrazio me stesso" è poco radiofonico.} \footnote{e così abbiamo scoperto anche le note a piè di pagina.}
per aver aggiornato il modello e scritto la documentazione per installare tutti i programmi per lavorare al documento.

Ringrazio Gemma Pia Romeo per aver fornito la sua tesi come punto di partenza per gli esempi presenti in queste pagine.




\bibliographystyle{apalike}
\bibliography{biblio}

\end{document} 