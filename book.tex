\documentclass[envcountsame,envcountchap]{svmono}
% Import fontspec package for font management
\usepackage{fontspec}

% Set main font (example using "Times New Roman")
%\setmainfont{Baskerville}

\usepackage{parskip}
\usepackage{makeidx}    % allows index generation
\usepackage{graphicx}   % standard LaTeX graphics tool when including figure files
\usepackage{subcaption} % per più figure sulla stessa riga
\usepackage{wrapfig}
\usepackage{multicol}
\usepackage{booktabs}
\usepackage{latexsym}
\usepackage[italian]{babel}
\usepackage{blindtext}
\usepackage{xcolor}
\usepackage{multirow}
\usepackage{multicol}
\usepackage{float}
\usepackage{minted}
\usepackage[hyphens]{url}
\usepackage{acronym}
\usepackage{natbib}
\usepackage{array}
\usepackage{indentfirst}
\usepackage{microtype}
\usepackage{lscape}
\usepackage{amsmath}
\usepackage[normalem]{ulem} % testo barrato
\usepackage{hyperref} % serve per inserire link nel testo, 
% utile solo per documenti non destinati alla stampa


\setlength{\textwidth}{12.7cm}
\setlength{\textheight}{20.0cm}
\setlength{\oddsidemargin}{3.50cm}
\setlength {\evensidemargin}{-0.3cm}  %foglio a destra 0.06
\setlength {\topmargin}{-1cm}
\setlength{\parindent}{15pt}
\setcounter{tocdepth}{5}  %serve per mettere nell'indice i sotto-sotto paragrafi
\setcounter{secnumdepth}{5} %serve per numerare i sotto-sotto paragrafi

\renewcommand\definitionname{Definizione}
\renewcommand\theoremname{Teorema}
\renewcommand\propositionname{Proposizione}
\renewcommand\corollaryname{Corollario}
\renewcommand\examplename{Esempio}




\def\punto{\hspace*{\fill}\Box}


\makeindex             % used for the subject index
                       % please use the style svind.ist with
                       % your makeindex program

%%%%%%%%%%%%%%%%%%%%%%%%%%%%%%%%%%%%%%%%%%%%%%%%%%%%%%%%%%%%%%%%%%%%%
\date{}
\begin{document}

%\maketitle


\pagenumbering{Roman}
%\setcounter{page}{5}

\frontmatter

\begin{titlepage}

    \begin{center}

    \large{\bf Università degli Studi Mediterranea di Reggio Calabria}

    \vspace*{1mm}

    \large{Dipartimento di Ingegneria Civile, Energia, Ambiente e Materiali}

    \vspace*{1mm}

    \normalsize{Corso di Laurea in Ingegneria ...}

    \vspace*{1mm}

    \hspace*{-0mm}

    \rule{125mm}{.2mm}  %{lunghezza}{spessore}


    \vspace{18mm}

    \begin{figure}[h!]
        \centerline{\includegraphics[width=3cm]{logounirc.png}}
    \end{figure}

    \vspace{5mm}

    \textbf{Tesi di Laurea}

    \vspace{5mm}

    %% TITOLO DELLA TESI
    \large{\bf TITOLO TESI DI LAUREA}

    \vspace{22mm}

    \begin{tabular}{lcl}
        {\large Relatore} & \ \hskip 2.2cm \ & {\large Candidato} \\
        \ & \ & \ \\
        {Alessandro Campolo} &               & {Gabriele Chirico}\\
        \ & \ & \ \\
        {\large Correlatore} &               & \\ %commentare se assente
        \ & \ & \ \\
        {Alessandro Campolo} &               & \\ %commentare se assente
        \\
    \end{tabular}

    \rule{125mm}{.2mm}

    \textbf{Anno Accademico 2023-2024}
    \end{center}

\end{titlepage}

\newpage
\newpage
\cleardoublepage
\thispagestyle{empty}
\vspace*{\stretch{1}}
\begin{flushright}
\itshape
<DEDICA>
\end{flushright}
\vspace{\stretch{2}}
\cleardoublepage

\newpage

\thispagestyle{empty}

% questo aggiunge l'indice al documento
\tableofcontents

\listoffigures
\listoftables



\pagenumbering{arabic} \setcounter{page}{9}

\chapter*{Indice degli acronimi} \label{acronimi}
\markboth{Indice degli acronimi}{Indice degli acronimi}
\begin{acronym}[WYSIWYM]
    % gli acronimi devono essere ordinati manualmente
    \acro{DICEAM}{Dipartimento di Ingegneria Civile, dell'Energia, dell'Ambiente e dei Materiali}
    \acro{DIIES}{Dipartimento di Ingegneria dell'Informazione, delle Infrastrutture e dell'Energia Sostenibile}
    \acro{DiGiES}{Dipartimento di Giurisprudenza, Economia e Scienze Umane}
    \acro{dArTe}{Dipartimento di Architettura e Territorio}
    \acro{PAU}{Dipartimento di Patrimonio, Architettura e Urbanistica}
\end{acronym}


\mainmatter%%%%%%%%%%%%%%%%%%%%%%%%%%%%%%%%%%%%%%%%%%%%%%%%%%%%%%%

% il motivo dei seguenti tre comandi è di creare un nuovo capitolo regolare, ma senza numero
\chapter*{Introduzione} \label{introduzione}
% l'asterisco crea un capitolo non numerato, non presente nell'indice
% e senza titolo in intestazione e piè di pagina
\addcontentsline{toc}{chapter}{Introduzione}
% addcontentsline aggiunge una voce all'indice
\markboth{Introduzione}{Introduzione}
% markboth imposta intestazione e piè di pagina

\vspace{2cm}

\hfill \textit{Introduzione al documento}

\vspace{0.5cm}

%contenuto dell'introduzione

\chapter{CAPITOLO 1} \label{Cap.1}

\vspace{2cm}

\begin{flushright}
\textit{TeX}
\end{flushright}

\vspace{0.5cm}


\section{SEZIONE}\label{sottosezione}


\subsection{SOTTO SEZIONE}

\chapter{CAPITOLO 2} \label{Cap.2}

\vspace{2cm}

\begin{flushright}
\textit{}
\end{flushright}

\vspace{0.5cm}


\chapter{Esempi}  \label{Cap.3}

\vspace{2cm}

\begin{flushright}
 \textit{Altri esempi di scrittura in \LaTeX\\ che non è stato possibile inserire nei capitoli precedenti.}
\end{flushright}

\vspace{0.5cm}

\section{minted}
Uso del pacchetto {\tt minted} per la visualizzazione di codice Python:
\begin{minted}{python}
import numpy as np
    
def incmatrix(genl1,genl2):
    m = len(genl1)
    n = len(genl2)
    M = None #to become the incidence matrix
    VT = np.zeros((n*m,1), int)  #dummy variable
    
    #compute the bitwise xor matrix
    M1 = bitxormatrix(genl1)
    M2 = np.triu(bitxormatrix(genl2),1) 

    for i in range(m-1):
        for j in range(i+1, m):
            [r,c] = np.where(M2 == M1[i,j])
            for k in range(len(r)):
                VT[(i)*n + r[k]] = 1;
                VT[(i)*n + c[k]] = 1;
                VT[(j)*n + r[k]] = 1;
                VT[(j)*n + c[k]] = 1;
                
                if M is None:
                    M = np.copy(VT)
                else:
                    M = np.concatenate((M, VT), 1)
                
                VT = np.zeros((n*m,1), int)
    
    return M
\end{minted}


\section{Testo multicolonna}

\begin{multicols}{2}
    \textbf{Colonna 1}\\
    Testo della prima colonna.\\
    Lorem ipsum dolor sit amet, consectetuer adipiscing elit. Etiam lobortis facilisis sem.
    \columnbreak\\
    \textbf{Colonna 2}\\
    Testo della seconda colonna.\\
    Lorem ipsum dolor sit amet, consectetuer adipiscing elit. Etiam lobortis facilisis sem. Nullam nec mi et neque pharetra sollicitudin. Praesent imperdiet mi nec ante. Donec ullamcorper, felis non sodales commodo, lectus velit ultrices augue, a dignissim nibh lectus placerat pede.
\end{multicols}

\newpage
\section{Spazi e ritorni}
{\tt \textbackslash newpage} termina la pagina attuale e inizia una nuova pagina.

Segue uno spazio vuoto di 2cm.

\vspace*{2cm}

Per forzare uno o più spazi bisogna usare uno o più backslash, ognuno seguito da uno spazio.\\
Testo normale\\
Testo \ con uno spazio in più\\
Testo \ \ con due spazi in più\\
Testo \ \ \ con tre spazi in più\\

Per forzare \LaTeX\ ad inserire una nuova riga vuota, inserire uno spazio e poi andare a capo.\\
\begin{verbatim}
\ \\
\end{verbatim}
Per scrivere questa sezione è stato usato il blocco {\tt verbatim}, il testo al suo interno non viene
interpretato in alcun modo.



Un elenco di elenchi??

Roba degna del più famoso elencatore d'Italia.
\begin{itemize}
    \item Dell'aria
    \item Dell'acqua
    \item Dei fiumi
    \item Dei mari
    \item dei nostri boschi
    \item delle nostre montagne
    \item dei nostri ghiacciai
    \begin{itemize}
        \item puliti
        \item riciclabili
        \item rinnovabili
        \item biodegradabili
    \end{itemize} 
\end{itemize}





\newpage
\begin{landscape}
    \section{Testo orizzontale}
    landscape ruota il contenuto della pagina in orizzontale.
    
    Per scrivere un'equazione su \LaTeX si può ricorrere a molteplici modi:
    \begin{itemize}
        \item $F=ma$
        \item Per centrare l'equazione:
        \[E=mc^2\]
    \end{itemize}
    Se si vuole numerare e centrare l'equazione si procede in questo modo:
    \begin{equation}
        F=ma 
    \end{equation}

\end{landscape}

\section{Immagini e tabelle}

Per inserire due immagini sotto la stessa didascalia bisogna seguire il seguente iter:

\begin{figure}
    \centering
    \subfloat[][\emph{Immagine 1}]
    {\includegraphics[width=.5\textwidth]{images/texlive/win/1_smart_screen.png}} \quad
    \subfloat[][\emph{Immagine 2}]
    {\includegraphics[width=.5\textwidth]{images/texlive/win/2_smart_screen_2.png}} \\
\caption{Due immagini di prova}
\label{struttura}
\end{figure}

\begin{table}[!ht]
    \centering
    \caption{ESEMPIO}
    \begin{tabular}{llll}
    \hline
        ESEMPIO & 1 & 2 & 3 \\ \hline
        A & A1 & A2 & A3 \\ 
        B & B1 & B2 & B3 \\ 
        C & C1 & C2 & C3 \\ \hline
    \end{tabular}
    \label{ESEMPIO}
\end{table}

Approfondimento sul posizionamento delle immagini \citep{forceFigurePlacement}

\section{Scorciatoie e consigli}

Ecco alcune indicazioni per ottimizzare al meglio il tuo documento:

\begin{itemize}
    \item Se il documento è molto lungo e ci metti tanto tempo
    per risalire al codice ma hai il pdf aperto, 
    fai Ctrl+click (pulsante sx del mouse) sul testo che cerchi nel pdf.
    \item se vuoi scrivere una formula matematica in grassetto, con all'interno simboli, vanno utilizzati
    i comandi $mathbf$ e $boldsymbol$. Ecco un esempio: $\mathbf{a=\boldsymbol{\Gamma}^{-1}u}$.
    \item Per costruire tabelle in modo più rapido utilizza il seguente link: \url{https://tableconvert.com/it/excel-to-latex}
\end{itemize}

    


\chapter*{Conclusioni}
\addcontentsline{toc}{chapter}{Conclusioni}
\markboth{Conclusioni}{Conclusioni}

\vspace{2cm}

\begin{flushright}
 \textit{<SINTESI DELLE CONCLUSIONI>}
\end{flushright}

\vspace{0.5cm}

Le conclusioni funzionano come tutti gli altri capitoli.

Come l'introduzione questo capitolo non ha il numero.

\section*{Sezione delle conclusioni}
La sezione con l'asterisco non ha il numero.

\chapter*{Ringraziamenti}
\addcontentsline{toc}{chapter}{Ringraziamenti}
\markboth{Ringraziamenti}{Ringraziamenti}





\bibliographystyle{apalike}
\bibliography{biblio}

\end{document} 