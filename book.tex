\documentclass[envcountsame,envcountchap]{svmono}
% Import fontspec package for font management
\usepackage{fontspec}

% Set main font (example using "Times New Roman")
%\setmainfont{Baskerville}

\usepackage{parskip}
\usepackage{makeidx}    % allows index generation
\usepackage{graphicx}   % standard LaTeX graphics tool when including figure files
\usepackage{multicol}
\usepackage{booktabs}
\usepackage{latexsym}
\usepackage[italian]{babel}
\usepackage{blindtext}
\usepackage{xcolor}
\usepackage{multirow}
\usepackage{multicol}
\usepackage{float}
\usepackage{minted}
\usepackage[hyphens]{url}
\usepackage{acronym}
\usepackage{natbib}
\usepackage{array}
\usepackage{lscape}
\usepackage{hyperref} % serve per inserire link nel testo, 
% utile solo per documenti non destinati alla stampa


\setlength{\textwidth}{12.7cm}
\setlength{\textheight}{20.0cm}
\setlength{\oddsidemargin}{3.50cm}
\setlength {\evensidemargin}{-0.3cm}  %foglio a destra 0.06
\setlength {\topmargin}{-1cm}

\renewcommand\definitionname{Definizione}
\renewcommand\theoremname{Teorema}
\renewcommand\propositionname{Proposizione}
\renewcommand\corollaryname{Corollario}
\renewcommand\examplename{Esempio}




\def\punto{\hspace*{\fill}\Box}


\makeindex             % used for the subject index
                       % please use the style svind.ist with
                       % your makeindex program

%%%%%%%%%%%%%%%%%%%%%%%%%%%%%%%%%%%%%%%%%%%%%%%%%%%%%%%%%%%%%%%%%%%%%
\date{}
\begin{document}

%\maketitle


\pagenumbering{Roman}
%\setcounter{page}{5}

\frontmatter

\begin{titlepage}

    \begin{center}

    \large{\bf Università degli Studi Mediterranea di Reggio Calabria}

    \vspace*{1mm}

    \large{Dipartimento di Ingegneria Civile, Energia, Ambiente e Materiali}

    \vspace*{1mm}

    \normalsize{Corso di Laurea in Ingegneria Industriale}

    \vspace*{1mm}

    \hspace*{-0mm}

    \rule{125mm}{.2mm}  %{lunghezza}{spessore}


    \vspace{18mm}

    \begin{figure}[h!]
        \centerline{\includegraphics[width=3cm]{logounirc.png}}
    \end{figure}

    \vspace{5mm}

    \textbf{Tesi di Laurea}

    \vspace{5mm}

    %% TITOLO DELLA TESI
    \large{\bf Creazione di un modello \LaTeX\ aggiornato e semplificato per tesi di laurea.}

    \vspace{22mm}

    \begin{tabular}{lcl}
        {\large Relatore} & \ \hskip 2.2cm \ & {\large Candidato} \\
        \ & \ & \ \\
        {Alessandro Campolo} &                  & {Alessandro Campolo}\\
        \ & \ & \ \\
        {\large Correlatore} &               & \\ %commentare se assente
        \ & \ & \ \\
        {Alessandro Campolo} &               & \\ %commentare se assente
        \\
    \end{tabular}

    \rule{125mm}{.2mm}

    \textbf{Anno Accademico 2023-2024}
    \end{center}

\end{titlepage}

\newpage
\newpage
\cleardoublepage
\thispagestyle{empty}
\vspace*{\stretch{1}}
\begin{flushright}
\itshape
<DEDICA>
\end{flushright}
\vspace{\stretch{2}}
\cleardoublepage

\newpage

\thispagestyle{empty}

% questo aggiunge l'indice al documento
\tableofcontents

\listoffigures



\pagenumbering{arabic} \setcounter{page}{7}

\chapter*{Indice degli acronimi}
\markboth{Indice degli acronimi}{Indice degli acronimi}
\begin{acronym}[WYSIWYM]
    % gli acronimi devono essere ordinati manualmente
    \acro{DICEAM}{Dipartimento di Ingegneria Civile, dell'Energia, dell'Ambiente e dei Materiali}
    \acro{DIIES}{Dipartimento di Ingegneria dell'Informazione, delle Infrastrutture e dell'Energia Sostenibile}
    \acro{DiGiES}{Dipartimento di Giurisprudenza, Economia e Scienze Umane}
    \acro{dArTe}{Dipartimento di Architettura e Territorio}
    \acro{PAU}{Dipartimento di Patrimonio, Architettura e Urbanistica}
\end{acronym}



% il motivo dei seguenti tre comandi è di creare un nuovo capitolo regolare, ma senza numero
\chapter*{Introduzione} \label{introduzione}
% l'asterisco crea un capitolo non numerato, non presente nell'indice
% e senza titolo in intestazione e piè di pagina
\addcontentsline{toc}{chapter}{Introduzione}
% addcontentsline aggiunge una voce all'indice
\markboth{Introduzione}{Introduzione}
% markboth imposta intestazione e piè di pagina

\vspace{2cm}

\hfill \textit{Introduzione al documento}

\vspace{0.5cm}


Il presente lavoro intende essere un modello di tesi del Dipartimento DICEAM 
(per l'esempio è stato usato il corso di Ingegneria Industriale).

Nel corso della trattazione verranno illustrati:
\begin{itemize}
    \item Installazione e configurazione di:
        \begin{itemize}
            \item TeX Live
            \item Visual Studio Code
            \item Estensione Latex Workshop per Visual Studio Code
        \end{itemize}
    \item Introduzione ai sistemi di controllo versione:
        \begin{itemize}
            \item git
            \item GitHub
            \item Fork e modifica di una repository
        \end{itemize}
    \item Esempi di elementi di \LaTeX\ come:
        \begin{itemize}
            \item immagini
            \item elenchi
            \item bibliografia
            \item tabelle
            \item spazi, righe e pagine
        \end{itemize}
\end{itemize}
  



\mainmatter%%%%%%%%%%%%%%%%%%%%%%%%%%%%%%%%%%%%%%%%%%%%%%%%%%%%%%%


\chapter{Preparazione dell'ambiente} \label{Cap.1}

\vspace{2cm}

\begin{flushright}
\textit{Installazione e configurazione del compilatore TeX Live e dell'editor Visual Studio Code}
\end{flushright}

\vspace{0.5cm}

L'installazione dei programmi verrà documentata per le tre principali piattaforme
\begin{itemize}
    \item Windows
    \item MacOS
    \item Linux e Unix-like
\end{itemize}

\section{TeX Live}
Essendo il processo più lungo, è consigliabile cominciare con l'installazione di TeX Live.

\subsection{Windows}
Per l'installazione su Windows scaricare l'installer da \url{https://mirror.ctan.org/systems/texlive/tlnet/install-tl-windows.exe}.

L'installazione dovrebbe andare a buon fine con le impostazioni di default.

\subsection{MacOS \citep{installMacTeX}}
Dopo il download d \url{https://mirror.ctan.org/systems/mac/mactex/MacTeX.pkg}, fare doppio clic per installarlo. Seguite le semplici istruzioni. L'installazione su un Macintosh recente richiede circa dieci minuti.

Il programma di installazione presenta:
\begin{itemize}
    \item una pagina di benvenuto
    \item una pagina ReadMe con ulteriori informazioni
    \item una pagina di licenza software
    \item una pagina finale, facendo clic sul pulsante “Installa” in questa pagina si avvia l'installazione vera e propria
\end{itemize}

La pagina finale presenta anche un pulsante “Personalizza”, che conduce a un pannello che consente agli utenti di decidere quali pezzi installare: Ghostscript, le applicazioni dell'interfaccia grafica e TeX Live stesso. La maggior parte degli utenti sceglierà l'installazione standard e ignorerà il pulsante “Personalizza”. Gli utenti che usano MacPorts o HomeBrew potrebbero preferire l'uso di Ghostscript fornito da questi progetti; in tal caso dovrebbero usare “Personalizza” per disabilitare l'installazione di Ghostscript.

Al termine dell'installazione, il programma di installazione riporta “Success”.


\subsubsection{Risoluzione dei problemi}
A volte, l'installatore visualizza una finestra di dialogo che dice “Verifica...” e poi l'installazione si blocca. In tutti i casi conosciuti, il riavvio del Macintosh risolve il problema. Dopo il riavvio, eseguire nuovamente l'installazione.

Se durante l'installazione vengono segnalati altri problemi, passare alla sezione “Errori di installazione”.

MacTeX scrive un collegamento simbolico /Library/TeX/texbin che punta indirettamente alla directory binaria di TeX Live. Configurare i programmi GUI per utilizzare questo collegamento. I programmi GUI forniti si configurano automaticamente.

Nel caso in cui la compilazione dei documenti dovesse dare un errore di tipo ENOENT, è necessario aggiungere il compilatore al PATH,
per farlo aprire un terminale e scrivere:
\begin{minted}{zsh}
    nano $HOME/.zshrc
\end{minted}
Nel file che si apre aggiungere alla fine la riga
\begin{minted}{zsh}
    export PATH="/usr/local/texlive/2024/bin/universal-darwin:$PATH"
\end{minted}
Per sicurezza verificare il percorso, perchè la cartella 2024 cambia in base alla versione, mentre la cartella universal-darwin cambia in base, non solo alla versione, ma anche all'architettura del processore.

Dopo questo passaggio è consigliabile chiudere COMPLETAMENTE Visual Studio Code dal menù in alto a sinistra e riaprirlo, se non dovesse risolvere, riavviare il computer.

Questa che hai appena letto è una sotto-sottosezione.

Non ti sono bastati 3 livelli?

\href{https://tex.stackexchange.com/questions/60209/how-to-add-an-extra-level-of-sections-with-headings-below-subsubsection}{Questo è un link per stackexchange, spiega come aggiungere altri livelli di sottosezioni.}

\subsection{Linux e Unix}
Per voi uomini coraggiosi che non avete paura di usare un terminale, propongo i comandi per l'installazione su distribuzioni Debian-based:
\begin{minted}{bash}
    sudo apt install texlive-science texlive-latex-extra latexmk \
        texlive-extra-utils texlive-publishers texlive-science
\end{minted}

\section{Visual Studio Code}
Per la scrittura si userà Visual Studio Code, editor multipiattaforma estensibile con numerosi plug-in.
D'ora in poi ci si riferirà ad esso col suo nome breve: vscode.

\subsection{Installazione}
\subsubsection{Windows, MacOS e distribuzioni Linux senza snap}
Per l'installazione su queste piattaforme si consiglia di seguire direttamente il sito del programma \url{https://code.visualstudio.com/download}.

\subsubsection{Distribuzioni con snap}
\begin{minted}{bash}
    sudo snap install code --classic
\end{minted}

\subsection{Configurazione}
Installato Visual Studio Code e completata la configurazione iniziale (opzionale),
la schermata presentata sarà simile a questa.
\begin{figure}[H]
    \centering
    \includegraphics[width=\linewidth]{images/vscode/vscode.png}
    \caption{Schermata iniziale di vscode}
    \label{schermata_iniziale_vscode}
\end{figure}
Cliccare su \includegraphics[height=5mm]{images/vscode/estensioni_icona.png}
per aprire il menu delle estensioni.
\begin{figure}[H]
    \centering
    \includegraphics[width=\linewidth]{images/vscode/vscode_latex_workshop.png}
    \caption{Schermata gestore estensioni}
    \label{schermata_gestore_estensioni}
\end{figure}
Nella barra di ricerca cercare Latex Workshop e premere su installa per avviare l'installazione.
Terminata l'installazione aprire il menu delle azioni rapide con Ctrl + Alt + P (Cmd + Alt + P su MacOS).
\begin{figure}[H]
    \centering
    \includegraphics[width=\linewidth]{images/vscode/vscode_user_settings.png}
    \caption{Menu azioni rapide}
    \label{menu_azioni_rapide}
\end{figure}
Cercare "Open User Settings (JSON)" come in figura \ref{menu_azioni_rapide} e selezionare la voce.
\begin{figure}[H]
    \centering
    \includegraphics[width=\linewidth]{images/vscode/vscode_settings_json.png}
    \caption{File JSON delle impostazioni}
    \label{json_impostazioni}
\end{figure}
Nella schermata che si apre incollare il contenuto del file my\_settings.json presente nella cartella del progetto.
È possibile eliminare il file dopo aver inserito il suo contenuto nel file settings.json di vscode.\\
ATTENZIONE: Il file che si sta modificando è un file JSON, in quanto tale richiede alcuni semplici accorgimenti di sintassi.
Se, come nella figura \ref{json_impostazioni}, sono già presenti altre righe, dopo l'ultima è necessario aggiungere una virgola prima di incollare il resto delle impostazioni.

Il risultato finale dovrebbe essere qualcosa di simile a \ref{json_impostazioni_nuovo}
\begin{figure}[H]
    \centering
    \includegraphics[width=\linewidth]{images/vscode/vscode_settings_json2.png}
    \caption{File JSON delle impostazioni dopo la modifica}
    \label{json_impostazioni_nuovo}
\end{figure}




% \section{È il momento delle figure}

% \LaTeX adora spostare le figure, quindi bisognerà un po' giocare col testo per trovare la posizione adatta.


% \begin{figure}[h]
%     \centering 
%     \includegraphics[width=0.2\linewidth]{logodiceam.png} 
%     \caption{Per comodità useremo il logo del dipartimento DICEAM.}
%     \label{figura_logo_diceam}
% \end{figure}

% In generale è consigliabile lasciar fare tutto al compilatore e poi referenziare la figura così \ref{figura_logo_diceam}. \citep{forceFigurePlacement}

% Se proprio è necessario posizionare l'immagine in una posizione precisa, usare [H] invece di [h].

% \section{Syntax highlighting}
% Se a qualcuno piacesse scrivere codice e magari renderlo leggibile, minted fa al caso suo.
% Altri stili: \\
% \url{https://www.overleaf.com/learn/latex/Code_Highlighting_with_minted}

% \begin{minted}{python}
% import numpy as np
    
% def incmatrix(genl1,genl2):
%     m = len(genl1)
%     n = len(genl2)
%     M = None #to become the incidence matrix
%     VT = np.zeros((n*m,1), int)  #dummy variable
    
%     #compute the bitwise xor matrix
%     M1 = bitxormatrix(genl1)
%     M2 = np.triu(bitxormatrix(genl2),1) 

%     for i in range(m-1):
%         for j in range(i+1, m):
%             [r,c] = np.where(M2 == M1[i,j])
%             for k in range(len(r)):
%                 VT[(i)*n + r[k]] = 1;
%                 VT[(i)*n + c[k]] = 1;
%                 VT[(j)*n + r[k]] = 1;
%                 VT[(j)*n + c[k]] = 1;
                
%                 if M is None:
%                     M = np.copy(VT)
%                 else:
%                     M = np.concatenate((M, VT), 1)
                
%                 VT = np.zeros((n*m,1), int)
    
%     return M
% \end{minted}

\chapter{Controllo versione} \label{Cap.2}

\vspace{2cm}

\begin{flushright}
\textit{Breve introduzione al sistema di controllo versione per sincronizzazione dei file del documento su server GitHub.}
\end{flushright}

\vspace{0.5cm}

La scrittura di un la








\chapter{<TERZO CAPITOLO>}  \label{Cap.3}

\vspace{2cm}

\begin{flushright}
 \textit{<<>>}
\end{flushright}

\vspace{0.5cm}



Testo multicolonna: 

\begin{multicols}{2}
    \textbf{Colonna 1}\\
    Testo della prima colonna.
\columnbreak\\
    \textbf{Colonna 2}\\
    Non lo scrivo nemmeno.
\end{multicols}


Signore e signori, sua maestà l'elenco numerato:
\begin{enumerate}
    \item Lorem
    \item Ipsum
    \item Dolor
    \item Sit
    \item Amet
    \item Sì ho avuto molta fantasia
\end{enumerate} 

%questo abominio serve per forzare latex a saltare una riga
\ \\

Per forzare uno o più spazi bisogna usare uno o più backslash, ognuno seguito da uno spazio.\\
Testo normale\\
Testo \ con uno spazio\\
Testo \ \ con due spazi\\
Testo \ \ \ con tre spazi

Un elenco di elenchi??

Roba degna del più forte elencatore d'Italia. \citep{elencatoreSeriale}
\begin{itemize}
    \item \textbf{A}: AAAA
    \begin{itemize}
        \item Aa
        \item Ab
        \item Ac
        \item Ad
        \item Ae
        \item Af
    \end{itemize} 
    \item \textbf{B}: BBBB
    \begin{itemize}
        \item Ba
        \item Bb
        \item Bc
        \item Bd
        \item Be
        \item Bf
    \end{itemize}
\end{itemize}

\newpage
newpage termina la pagina attuale e inizia una nuova pagina.

\vspace*{2cm}

Non avevamo ancora parlato delle tabelle.

Sono lunghe da trattare quindi lasciamo il duro compito alla documentazione di Overleaf.
\url{https://www.overleaf.com/learn/latex/Tables}

\begin{table}[h!]
\centering
\begin{tabular}{||c c c c||} 
    \hline
    Col1 & Col2 & Col2 & Col3 \\ [0.5ex] 
    \hline\hline
    1 & 6 & 87837 & 787 \\ 
    2 & 7 & 78 & 5415 \\
    3 & 545 & 778 & 7507 \\
    4 & 545 & 18744 & 7560 \\
    5 & 88 & 788 & 6344 \\ [1ex] 
    \hline
\end{tabular}
\caption{Table to test captions and labels.}
\label{table:1}
\end{table}


\newpage
\begin{landscape}
landscape mette il contenuto della pagina in orizzontale.
\end{landscape}
    


\chapter*{Conclusioni}
\addcontentsline{toc}{chapter}{Conclusioni}
\markboth{Conclusioni}{Conclusioni}

\vspace{2cm}

\begin{flushright}
 \textit{<SINTESI DELLE CONCLUSIONI>}
\end{flushright}

\vspace{0.5cm}

Le conclusioni funzionano come tutti gli altri capitroli.

Come l'introduzione questo capitolo non ha il numero.

\section*{Sezione delle conclusioni}
La sezione con l'asterisco non ha il numero.

\chapter*{Ringraziamenti}
\addcontentsline{toc}{chapter}{Ringraziamenti}
\markboth{Ringraziamenti}{Ringraziamenti}

Ringrazio il dipartimento DIIES per aver pubblicato un modello di tesi non troppo aggiornato.

Ringrazio Alessandro Campolo\footnote{``ringrazio me stesso" è poco radiofonico.} \footnote{e così abbiamo scoperto anche le note a piè di pagina.}
per aver aggiornato il modello e scritto la documentazione per installare tutti i programmi per lavorare al documento.

Ringrazio Gemma Pia Romeo per aver fornito la sua tesi come punto di partenza per gli esempi presenti in queste pagine.




\bibliographystyle{apalike}
\bibliography{biblio}

\end{document} 